% MRF Compiler Reference Manual
% Copyright (C) 2025, Shyamal Suhana Chandra

\documentclass[12pt,a4paper]{article}
\usepackage[utf8]{inputenc}
\usepackage{amsmath}
\usepackage{listings}
\usepackage{xcolor}
\usepackage{hyperref}
\usepackage{longtable}

\title{MRF Compiler Reference Manual}
\author{Shyamal Suhana Chandra}
\date{\today}

\begin{document}

\maketitle
\tableofcontents
\newpage

\section{Introduction}

This reference manual provides complete documentation for the MRF Compiler, including command-line options, input formats, API reference, and framework-specific details.

\section{Command-Line Interface}

\subsection{Basic Usage}

\verb|./mrf_compiler [options] [input_file] [output_file]|

\subsection{Options}

\begin{longtable}{|p{3cm}|p{10cm}|}
\hline
\textbf{Option} & \textbf{Description} \\
\hline
\endfirsthead
\hline
\textbf{Option} & \textbf{Description} \\
\hline
\endhead
\verb|-f, --framework| & Specify output framework (qasm, qiskit, cirq, pennylane, qsharp, braket, qulacs, tfq) \\
\hline
\verb|-a, --all| & Export to all supported frameworks \\
\hline
\verb|-h, --help| & Display help message \\
\hline
\end{longtable}

\section{Input Format}

\section{Output Formats}

\section{API Reference}

\section{Framework Details}

\section{Examples}

\section{Troubleshooting}

\end{document}
